\RequirePackage[l2tabu, orthodox]{nag} % Warn about outdated commands/packages.
% The report class uses some outdated commands, about which nag will complain.
% You can just ignore these warnings.

\documentclass[11pt, a4paper]{report} % Sets font and paper size.

%%% General formatting packages (order is important, so don't sort) %%%
\usepackage{amsmath} % More equation formatting.
\usepackage[english]{babel} % Language specific quirks.
\usepackage{epigraph}
\usepackage{booktabs} % Improved tables.
\usepackage{doi}
\usepackage[font=small]{caption} % Better caption formatting.
\usepackage{fancyhdr} % Modification of headers and footers.
\usepackage[T1]{fontenc} % Makes one unicode character of special input (e.g. ö).
\usepackage[margin=1.25in]{geometry} % Control page layout.
\usepackage{float} % More control over image positions.
\usepackage{graphicx} % Include graphics. Use '\graphicspath' to locate files in a different folder.
\usepackage[utf8]{inputenc} % Special characters (e.g. trema) can be entered directly: .tex file has to be saved using UTF-8 encoding.
\usepackage{lmodern} % Alternative font because 'fontenc' package does not work with default.
\usepackage{microtype} % Improves character spacing.
\usepackage{physics} % Provides physics macros such as Dirac notation.
\usepackage{tikz} % Draw diagrams and figures.
\usepackage{url} % Allow inclusion of urls in text.
\usepackage{siunitx} % SI unit formatting and scientific notation.
\usepackage{subcaption} % Allow subcaptions.
\usepackage[nottoc]{tocbibind} % Include references in table of contents.
\usepackage[colorinlistoftodos,textsize=tiny]{todonotes} % Add todo notes.
\usepackage[noabbrev]{cleveref} % Automate "equation (...)" reference. Use \cref.

%%% Additional options %%%
\pagestyle{fancy} % Set header style.
\setlength{\headheight}{14pt}

\setcounter{tocdepth}{3}
\setcounter{secnumdepth}{3}

%%% Personal Information %%%
\newcommand\TITLE{The rise of SkyNet}
\newcommand\THESISFORM{Master Thesis}
\newcommand\AUTHOR{Teun Zwart}
\newcommand\SUPERVISOR{Prof. dr. Jean-Sébastien Caux}
\newcommand\UNIVERSITYLOGO{} % Uncomment line below and add name of logo file.

\newcommand{\bea}{\begin{align}}
\newcommand{\ena}{\end{align}}


\begin{document}

\begin{titlepage}
	\begin{center}
		% \includegraphics[width=\textwidth]{\UNIVERSITYLOGO}
		\rule{\textwidth}{0.4mm}\\[0.5cm]
		\huge{\textbf{\TITLE}}
		\rule{\textwidth}{0.4mm}\\[0.5cm]
		\Large{\THESISFORM}\\[0.5cm]
		\begin{minipage}[t]{0.4\textwidth}
			\begin{flushleft}
				\large\emph{Author:}\\{\AUTHOR}
			\end{flushleft}
		\end{minipage}
		\begin{minipage}[t]{0.4\textwidth}
			\begin{flushright}
				\large\emph{Supervisor:}\\{\SUPERVISOR}
			\end{flushright}
		\end{minipage}
		\vfill
		\large \today\\
	\end{center}
\end{titlepage}

\newpage
\thispagestyle{empty}

\ 
\vspace{4cm}

\epigraph{\itshape Science is just magic that works.}{Kurt Vonnegut}

\epigraph{\itshape All sufficiently advanced technology is indistinguishable from magic.}{Arthur C. Clark}


\tableofcontents

\chapter{Introduction}

\chapter{The Bethe Ansatz}

The Bethe Ansatz allows for a way to solve certain one-dimensional systems exactly.
It was put forth by Bethe in 1931 to solve the Heisenberg model, refining and correcting the earlier work of Bloch \cite{Bethe1931}.

\section{The Lieb-Liniger model}

Here, to introduce and understand the Bethe Ansatz, we focus on the Lieb-Liniger model, finding its eigenfunctions and excitation spectrum.
The Lieb-Liniger model, first introduced by Lieb and Liniger in 1963 \cite{Lieb1963, Lieb1963a}, consists of bosons interaction through a delta-function potential, described by the Hamiltonian
\begin{equation}
	H = - \sum_{j=1}^{N} \frac{\partial^2}{\partial x_j^2} + 2c \sum_{j<l} \delta(x_j - x_l).
\end{equation}
Here \(c\) is the interaction strength of the model, \(\hbar=1\) and we set \(2m=1\).
In the limit \(c\to0\) particles behave as free bosons, while \(c\to\infty\) yields fermionic behaviour and is known as the Tonks-Girardeau model \cite{Lieb1963, Franchini2017}.

We will only solve the repulsive case (\(c > 0\)) here. \todo{Why not the attractive case, what does saturation for that mean? See super Tonks-Girardeau model.}

\subsection{The ground state}

We will start solving the Lieb-Liniger model in the two-particle case. This later readily generalizes to many particles.

\subsubsection{Two-particle states}

For two particles, the Hamiltonian becomes
\begin{equation}
	H =  - \frac{\partial^2}{\partial x_1^2} - \frac{\partial^2}{\partial x_2^2} + 2c \delta(x_1 - x_2).
\end{equation}
We can write this in a more convenient form without the delta-function by integrating the Schrödinger equation over a small interfal \([-\epsilon,\epsilon]\).
This yields the condition \cite{Lieb1963}\todo{Do this explicitly.}\footnote{This operation is relatively simple but does contain a few subleties. Let us therefore do it explicitly:
the Hamiltonian is
\begin{equation}
	\left[- \frac{\partial^2}{\partial x_1^2} - \frac{\partial^2}{\partial x_2^2} + 2c \delta(x_1 - x_2)\right] \psi(x_1, x_2) = E \psi(x_1,x_2).
\end{equation} 
Integrating \cite{Griffiths1993}
}
\begin{equation}\label{eq:lieblinigerboundary}
	\left[\frac{\partial}{\partial x_2} - \frac{\partial}{\partial x_1} - c\right] \psi(x_1, x_2)\bigg\rvert_{x_1 = x_2} = 0.
\end{equation}
Since particles only interact with each other when their positions coincide, this allows us to simplify the Schrödinger equation to
\begin{equation}\label{eq:lieblinigersimple}
	\left[- \frac{\partial^2}{\partial x_1^2} - \frac{\partial^2}{\partial x_2^2}\right] \psi(x_1, x_2) = E \psi(x_1,x_2),
\end{equation}
with \cref{eq:lieblinigerboundary} as a boundary condition \cite{Lieb1963}.
Since we deal with bosons, we also require the wavefunction to be symmetric: \(\psi(x_1,x_2) = \psi(x_2,x_1)\).

We can now conjecture a solution for the Lieb-Liniger model based on \cref{eq:lieblinigersimple}.
This equation can be solved in terms of a superposition of plane waves:
\begin{equation}
	\psi(x_1,x_2) = A_{12} e^{i(\lambda_1x_1 + \lambda_2 x_2)} + A_{21} e^{i(\lambda_2 x_1 + \lambda_1 x_2)},
\end{equation}
where the \(\lambda\)'s are pseudo-momenta, called so because they are unobservable \cite{Franchini2017}.
The \(A\)'s are the amplitudes of the planewaves making up the wavefunction.
Since \cref{eq:lieblinigerboundary} has to be fulfilled we require the amplitudes be related to each other by
\begin{align}
	\frac{A_{12}}{A_{21}} = -\frac{i(c-\lambda_1 - \lambda_2) }{i(c+\lambda_1 - \lambda_2)}.
\end{align}
The righthand side of this equation has modulus 1, meaning it can be written as a phase:
\begin{align}
	\frac{A_{12}}{A_{21}} = -e^{i\phi(\lambda_1-\lambda_2)},
\end{align}
where
\begin{align}
	\phi(\lambda_1-\lambda_2) = -2\arctan\left(\frac{\lambda_1-\lambda_2}{c}\right),
\end{align}
and we assume the argument of \(\phi\) to be real \cite{Lieb1963}.



\subsubsection{Many-particle states}
- uniqueness of the solutions
- Norms of the eigenstates: Gaudin matrices

\subsubsection{Quantization}

\subsection{The thermodynamic limit}

\subsection{Excited states}

\subsubsection{Type I excitations}

\subsubsection{Type II excitations}

\subsection{Thermodynamic properties}

\subsubsection{Various limit cases}






\chapter{The ABACUS algorithm}

-DSF\\
-Scanning over Hilbert space\\





\chapter{Machine learning in the ABACUS algorithm}

-use Monte Carlo to produce data sets for training

\chapter{Results}

\chapter{Discussion and conclusion}


\bibliographystyle{SciPost_bibstyle}
\bibliography{master_thesis_references.bib}

\end{document}
