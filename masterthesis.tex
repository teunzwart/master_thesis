\RequirePackage[l2tabu, orthodox]{nag} % Warn about outdated commands/packages.
% The report class uses some outdated commands, about which nag will complain.
% You can just ignore these warnings.

\documentclass[11pt, a4paper,draft]{report} % Sets font and paper size.

%%% General formatting packages (order is important, so don't sort) %%%
\usepackage{amsmath} % More equation formatting.
\usepackage{amssymb}
\usepackage[english]{babel} % Language specific quirks.
\usepackage{epigraph}
\usepackage{booktabs} % Improved tables.
\usepackage{doi}
\usepackage[font=small]{caption} % Better caption formatting.
\usepackage{fancyhdr} % Modification of headers and footers.
\usepackage[T1]{fontenc} % Makes one unicode character of special input (e.g. ö).
\usepackage[margin=1.25in]{geometry} % Control page layout.
\usepackage{float} % More control over image positions.
\usepackage{graphicx} % Include graphics. Use '\graphicspath' to locate files in a different folder.
\usepackage[utf8]{inputenc} % Special characters (e.g. trema) can be entered directly: .tex file has to be saved using UTF-8 encoding.
\usepackage{lmodern} % Alternative font because 'fontenc' package does not work with default.
\usepackage{microtype} % Improves character spacing.
\usepackage{physics} % Provides physics macros such as Dirac notation.
\usepackage{tikz} % Draw diagrams and figures.
\usepackage{url} % Allow inclusion of urls in text.
\usepackage{siunitx} % SI unit formatting and scientific notation.
\usepackage{subcaption} % Allow subcaptions.
\usepackage[nottoc]{tocbibind} % Include references in table of contents.
\usepackage[colorinlistoftodos,textsize=tiny]{todonotes} % Add todo notes.
\usepackage[noabbrev]{cleveref} % Automate "equation (...)" reference. Use \cref.

%%% Additional options %%%
\pagestyle{fancy} % Set header style.
\fancyhead[L]{\rightmark}
\fancyhead[R]{}
\setlength{\headheight}{14pt}

\setcounter{tocdepth}{3}
\setcounter{secnumdepth}{3}

%%% Personal Information %%%
\newcommand\TITLE{Thinking through problems thinking computers pose to future thinkers}
\newcommand\THESISFORM{Master Thesis}
\newcommand\AUTHOR{Teun Zwart}
\newcommand\SUPERVISOR{Prof. dr. Jean-Sébastien Caux}
\newcommand\UNIVERSITYLOGO{} % Uncomment line below and add name of logo file.

\newcommand{\bea}{\begin{align}}
\newcommand{\ena}{\end{align}}

\newcommand{\inversetruncc}{\mathcal{L}}
\newcommand{\kernel}{\mathcal{C}}



\begin{document}

\begin{titlepage}
	\begin{center}
		% \includegraphics[width=\textwidth]{\UNIVERSITYLOGO}
		\rule{\textwidth}{0.4mm}\\[0.5cm]
		\huge{\textbf{\TITLE}}
		\rule{\textwidth}{0.4mm}\\[0.5cm]
		\Large{\THESISFORM}\\[0.5cm]
		\begin{minipage}[t]{0.4\textwidth}
			\begin{flushleft}
				\large\emph{Author:}\\{\AUTHOR}
			\end{flushleft}
		\end{minipage}
		\begin{minipage}[t]{0.4\textwidth}
			\begin{flushright}
				\large\emph{Supervisor:}\\{\SUPERVISOR}
			\end{flushright}
		\end{minipage}
		\vfill
		\large \today\\
	\end{center}
\end{titlepage}

\newpage
\thispagestyle{empty}

\ 
\vspace{4cm}

\epigraph{\itshape Science is just magic that works.}{Kurt Vonnegut}

\epigraph{\itshape All sufficiently advanced technology is indistinguishable from magic.}{Arthur C. Clark}


\tableofcontents

\chapter{Introduction}

\chapter{The Bethe Ansatz}

The Bethe Ansatz allows for a way to solve certain one-dimensional systems exactly.
It was put forth by Bethe in 1931 to solve the Heisenberg model, refining and correcting the earlier work of Bloch \cite{Bethe1931}.

\section{The Lieb-Liniger model}

Here, to introduce and understand the Bethe Ansatz, we focus on the Lieb-Liniger model, finding its eigenfunctions and excitation spectrum.
The Lieb-Liniger model, first introduced by Lieb and Liniger in 1963 \cite{Lieb1963, Lieb1963a}, consists of bosons interaction through a delta-function potential, described by the Hamiltonian
\begin{equation}
	H = - \sum_{j=1}^{N} \frac{\partial^2}{\partial x_j^2} + 2c \sum_{j<l} \delta(x_j - x_l).
\end{equation}
Here \(c\) is the interaction strength of the model, \(\hbar=1\) and we set \(2m=1\).
In the limit \(c\to0\) particles behave as free bosons, while \(c\to\infty\) yields fermionic behaviour and is known as the Tonks-Girardeau model \cite{Lieb1963, Franchini2017}.

We will only solve the repulsive case (\(c > 0\)) here. \todo{Why not the attractive case, what does saturation for that mean? See super Tonks-Girardeau model.}

\subsection{A solution for the Schrödinger equation}

We will start solving the Lieb-Liniger model in the two-particle case. This later readily generalizes to many particles.

For two particles, the Hamiltonian becomes
\begin{equation}
	H =  - \frac{\partial^2}{\partial x_1^2} - \frac{\partial^2}{\partial x_2^2} + 2c \delta(x_1 - x_2).
\end{equation}
We can write this in a more convenient form without the delta-function by integrating the Schrödinger equation over a small interfal \([-\epsilon,\epsilon]\).
This yields the condition \cite{Lieb1963}\todo{Do this explicitly.}\footnote{This operation is relatively simple but does contain a few subtleties. Let us therefore do it explicitly:
the Schrödinger equation is
\begin{equation}
	\left[- \frac{\partial^2}{\partial x_1^2} - \frac{\partial^2}{\partial x_2^2} + 2c \delta(x_1 - x_2)\right] \psi(x_1, x_2) = E \psi(x_1,x_2).
\end{equation} 
Integrating \cite{Griffiths1993}
}
\begin{equation}\label{eq:lieblinigerboundary}
	\left[\frac{\partial}{\partial x_2} - \frac{\partial}{\partial x_1} - c\right] \psi(x_1, x_2)\bigg\rvert_{x_1 = x_2} = 0.
\end{equation}
Since particles only interact with each other when their positions coincide, this allows us to simplify the Schrödinger equation to
\begin{equation}\label{eq:lieblinigersimple}
	\left[- \frac{\partial^2}{\partial x_1^2} - \frac{\partial^2}{\partial x_2^2}\right] \psi(x_1, x_2) = E \psi(x_1,x_2),
\end{equation}
with \cref{eq:lieblinigerboundary} as a boundary condition \cite{Lieb1963}.
Since we are dealing with bosons, we also require the wavefunction to be symmetric: \(\psi(x_1,x_2) = \psi(x_2,x_1)\).

We can now conjecture a solution for the Lieb-Liniger model based on \cref{eq:lieblinigersimple}.
This equation can be solved in terms of a superposition of plane waves:
\begin{equation}
	\psi(x_1,x_2) = A_{12} e^{i(\lambda_1x_1 + \lambda_2 x_2)} + A_{21} e^{i(\lambda_2 x_1 + \lambda_1 x_2)},
\end{equation}
where the \(\lambda\)'s are pseudo-momenta, so-called because they are unobservable \cite{Franchini2017}.
The \(A\)'s are the amplitudes of the planewaves making up the wavefunction.
Since \cref{eq:lieblinigerboundary} has to be fulfilled we require the amplitudes be related to each other by
\begin{align}
	\frac{A_{12}}{A_{21}} = -\frac{c-i(\lambda_1 - \lambda_2) }{c+i(\lambda_1 - \lambda_2)}.
\end{align}
The right-hand side of this equation has modulus 1, meaning it can be written as a phase:
\begin{align}
	\frac{A_{12}}{A_{21}} = -e^{i\phi(\lambda_1-\lambda_2)},
\end{align}
where
\begin{align}
	\phi(\lambda_1-\lambda_2) = -2\arctan\left(\frac{\lambda_1-\lambda_2}{c}\right),
\end{align}
and we assume the argument of \(\phi\) to be real \cite{Lieb1963}.

We can now write the two-particle wavefunction by setting \(A_{21}=1\) (and ignoring normalization for the moment) to get\todo{Is the distribution of the phase allowed because arbitrary phases don't count?}
\begin{equation}
	\psi(x_1,x_2) = - e^{-i\phi(\lambda_1-\lambda_2)+i(\lambda_1x_1 + \lambda_2 x_2)} + e^{i(\lambda_2 x_1 + \lambda_1 x_2)}
\end{equation}

~\\
- full wavefunction for this case\\
- many particle state\\

Going to the many-particle state is now a fairly straightforward generalization of the two-particle case.
The wavefunction in this case becomes \cite{Franchini2017}
\begin{equation}
	\psi(x_1,\ldots,x_N) = \sum_{\mathcal{P}} A(\mathcal{P}) e^{i\sum_j \lambda_{\mathcal{P}_j} x_j}.
\end{equation}

\subsubsection{Quantization}

- two particles \\
- many particles\\
- Bethe equations\\
- uniqueness of the solutions for the quantum numbers\\
- Norms of the eigenstates: Gaudin matrices
- ground state

\subsection{The thermodynamic limit}

\subsubsection{Type I excitations}

\subsubsection{Type II excitations}

\subsection{Thermodynamic properties}

\subsubsection{Various limit cases}






\chapter{The ABACUS algorithm}

-DSF\\
-Scanning over Hilbert space\\
-current implementation of finding rapidities\\





\chapter{Machine learning in the ABACUS algorithm}

-use Monte Carlo to produce data sets for training\\
-neural networks\\
--basics: training, validation, testing\\
--feedforward\\
--convolutional\\
-- continuous, online learning


\section{Summing 1/L terms}

When updating we have to take care to not update in place but rather update into a new list, since otherwise we will use a different set of lambdas to calculate each new lambda.

For the displacement of the particle-like excitations we get the expression
\begin{equation}
	D_p(\lambda, \lambda_p) = - \int_{\lambda_p}^{\textrm{sgn}(\lambda_p)\infty} \dd \lambda'(1 + \inversetruncc^{(F)}) * \left(\theta(\lambda_F-\lvert\lambda\rvert) \kernel(\lambda-\lambda')\right) 
\end{equation}
or
\begin{equation}
	D_p(\lambda, \lambda_p) = - \int_{\lambda_p}^{\textrm{sgn}(\lambda_p)\infty} \dd \lambda' \int_{\lambda_F}^{\lambda_F} \dd  \lambda'' \left(\delta(\lambda-\lambda''\right) + \inversetruncc^{(F)}) \kernel(\lambda-\lambda')
\end{equation}


\newpage

\section{Displacement functions for excitations}
We now find the displacement function for particle-like (type I) and hole-like (type~II) excitations, first for the ground state and then for excited states.

\subsection{The groundstate}
\subsubsection{Type I}
For type I excitations the ground state displacement function is 
\begin{equation}\label{eq:particledisplacement}
	D_p(\lambda, \lambda_p) = - \int_{\lambda_p}^{\textrm{sgn}(\lambda_p)\infty} \dd \lambda' \int_{-\lambda_F}^{\lambda_F} \dd  \lambda'' \left(\delta(\lambda-\lambda'') + \inversetruncc^{(F)}(\lambda,\lambda'') \right)\kernel(\lambda''-\lambda'),
\end{equation}
valid for all \(\lambda_p\) (if \(\lvert \lambda_p \rvert\ > \lambda_F\)) and all \(\lambda\) \cite{tofind}.

We can rewrite this in a more explicit way.
Lets first focus on the part involving the \(\delta\)-function, which we can rewrite, using a Heaviside step function, as
\begin{align}
	- \int_{\lambda_p}^{\textrm{sgn}(\lambda_p)\infty} &\dd \lambda' \int_{-\lambda_F}^{\lambda_F} \dd  \lambda'' \delta(\lambda-\lambda'') \kernel(\lambda''-\lambda') \\
	&=- \int_{\lambda_p}^{\textrm{sgn}(\lambda_p)\infty} \dd \lambda' \int_{-\infty}^{\infty} \dd \lambda'' \theta(\lambda_F - \lvert \lambda'' \rvert)  \delta(\lambda-\lambda'') \kernel(\lambda''-\lambda')\\
	&= - \theta(\lambda_F - \lvert \lambda \rvert) \int_{\lambda_p}^{\textrm{sgn}(\lambda_p)\infty} \dd \lambda'     \kernel(\lambda-\lambda').
\end{align}
We change variables \(a=\lambda-\lambda'\):
\begin{equation}\label{eq:integratedkerneldeltapart}
	- \theta(\lambda_F - \lvert \lambda \rvert) \int_{\lambda_p}^{\textrm{sgn}(\lambda_p)\infty} \dd \lambda'     \kernel(\lambda-\lambda')=
	 \theta(\lambda_F - \lvert \lambda \rvert) \int_{\lambda-\lambda_p}^{\lambda - \textrm{sgn}(\lambda_p)\infty} \dd a\,\kernel(a).
\end{equation}
We assume \(\lambda\) to be finite, meaning the upper bound of the integral reduces to \(-\textrm{sgn}(\lambda_p)\).

\Cref{eq:integratedkerneldeltapart} is 0 if \(\lvert \lambda\rvert > \lambda_F\), so the only \(\lambda\)'s we have to consider in the integral are \(\lvert\lambda\rvert< \lambda_F\), but from the definition of the displacement we know \(\lvert\lambda_p\rvert >\lambda_F\), meaning \(\lvert\lambda_p\rvert > \lvert\lambda\rvert\) for all \(\lambda\) in the integral.
Therefore \(\lambda-\lambda_p\) always has the opposite sign from~\(\lambda_p\).
Thus we can rewrite \cref{eq:integratedkerneldeltapart} as
\begin{align}
	\theta(\lambda_F - \lvert \lambda \rvert) \int_{\lambda-\lambda_p}^{\lambda - \textrm{sgn}(\lambda_p)\infty} \dd a\,\kernel(a) &=
	\theta(\lambda_F - \lvert \lambda \rvert) \int_{-\textrm{sgn}(\lambda_p)\lvert\lambda-\lambda_p\rvert}^{ - \textrm{sgn}(\lambda_p)\infty} \dd a\,\kernel(a).
\end{align}

Since the kernel is symmetric, for the value of the integral it does not matter whether \(\mathrm{sgn}(\lambda_p)\) is \(+1\) or \(-1\).
It only adds a sign-function in front of the integral:
\begin{align}
	\theta(\lambda_F - \lvert \lambda \rvert) \int_{-\textrm{sgn}(\lambda_p)\lvert\lambda-\lambda_p\rvert}^{ - \textrm{sgn}(\lambda_p)\infty} \dd a\,\kernel(a) &= - \textrm{sgn}(\lambda_p) \theta(\lambda_F - \lvert \lambda \rvert) \int_{\lvert\lambda-\lambda_p\rvert}^{\infty} \dd a\,\kernel(a)\\
	&= - \frac{c}{\pi} \textrm{sgn}(\lambda_p) \theta(\lambda_F - \lvert \lambda \rvert)  \int_{\lvert\lambda-\lambda_p\rvert}^{\infty} \dd a\,\frac{1}{c^2 + a^2}\\
	&= - \frac{1}{2\pi} \textrm{sgn}(\lambda_p) \theta(\lambda_F - \lvert \lambda \rvert)  \left( \pi - 2\atan(\frac{\lvert \lambda - \lambda_p\rvert}{c})\right)\label{eq:dpdeltanonsimplified}\\
	&= - \frac{1}{\pi} \textrm{sgn}(\lambda_p) \theta(\lambda_F - \lvert \lambda \rvert)  \atan\left(\frac{c}{\lvert\lambda-\lambda_p\rvert}\right),
\end{align}
where the last equality is only valid since the argument of the arctangent is strictly positive.

It is important to remark that the above expression would not be valid without the Heaviside function, since the integral is only applicable when \(\lvert\lambda\rvert< \lambda_F\).

The second part of \cref{eq:particledisplacement}, which involves \(\inversetruncc^{(F)}(\lambda,\lambda')\), can't be rewritten in closed form, since there is no closed form solution for \(\inversetruncc^{(F)}(\lambda,\lambda')\).
We can however rewrite it to get rid of the improper integral, which will help in numerical evaluation of the expression.
We have
\begin{align}
	- \int_{\lambda_p}^{\textrm{sgn}(\lambda_p)\infty} &\dd \lambda' \int_{-\lambda_F}^{\lambda_F} \dd \lambda''  \inversetruncc^{(F)}(\lambda,\lambda'') \kernel(\lambda''-\lambda')\\
	&= -  \int_{-\lambda_F}^{\lambda_F} \dd \lambda''   \inversetruncc^{(F)}(\lambda,\lambda'') \left[\int_{\lambda_p}^{\textrm{sgn}(\lambda_p)\infty} \dd \lambda'\kernel(\lambda''-\lambda') \right]\\
	&=   \int_{-\lambda_F}^{\lambda_F} \dd \lambda''   \inversetruncc^{(F)}(\lambda,\lambda'') \left[\int_{\lambda''-\lambda_p}^{\lambda''-\textrm{sgn}(\lambda_p)\infty} \dd a\,\kernel(a) \right]\\
	&= -\textrm{sgn}(\lambda_p) \int_{-\lambda_F}^{\lambda_F} \dd \lambda''   \inversetruncc^{(F)}(\lambda,\lambda'') \left[\int_{\lvert\lambda''-\lambda_p\rvert}^{\infty} \dd a\,\kernel(a) \right]\\
	&= -\frac{1}{\pi}\textrm{sgn}(\lambda_p) \int_{-\lambda_F}^{\lambda_F} \dd \lambda''   \inversetruncc^{(F)}(\lambda,\lambda'') \left[\frac{\pi}{2} - \atan(\frac{\lvert\lambda''-\lambda_p\rvert}{c})\right]\\
	&= -\frac{1}{\pi}\textrm{sgn}(\lambda_p) \int_{-\lambda_F}^{\lambda_F} \dd \lambda''   \inversetruncc^{(F)}(\lambda,\lambda'') \atan(\frac{c}{\lvert\lambda''-\lambda_p\rvert}),
\end{align}
where, as before, we argued that \(\lvert\lambda''\rvert < \lvert\lambda_p\rvert\) and thus that \(\lambda'' - \lambda_p\) always has the opposite sign from \(\lambda_p\).

Taking these results together, the displacement function for type I excitations becomes 
\begin{multline}
	D_p(\lambda, \lambda_p) = - \frac{1}{\pi} \textrm{sgn}(\lambda_p) \theta(\lambda_F - \lvert \lambda \rvert)  \atan\left(\frac{c}{\lvert\lambda-\lambda_p\rvert}\right) \\-\frac{1}{\pi}\textrm{sgn}(\lambda_p) \int_{-\lambda_F}^{\lambda_F} \dd \lambda''   \inversetruncc^{(F)}(\lambda,\lambda'') \atan(\frac{c}{\lvert\lambda''-\lambda_p\rvert}),
\end{multline}

\subsubsection{Type II}
For type II excitations the idea is much the same as type I. The displacement is
\begin{equation}
	D_h(\lambda, \lambda_h) = - \int_{-\textrm{sgn}(\lambda_h)\infty}^{\lambda_h} \dd \lambda' \int_{-\lambda_F}^{\lambda_F} \dd \lambda'' \left(\delta(\lambda-\lambda'') + \inversetruncc^{(F)}(\lambda,\lambda'') \right)\kernel(\lambda''-\lambda'),
\end{equation}
valid for all \(\lambda_h\) (if \(\lvert \lambda_h \rvert\ < \lambda_F\)) and all \(\lambda\) \cite{tofind}.
The \(\delta\)-function part of this equation becomes
\begin{align}
	-\int_{-\textrm{sgn}(\lambda_h)\infty}^{\lambda_h} \dd \lambda' \int_{-\lambda_F}^{\lambda_F} \dd \lambda'' \delta(\lambda-\lambda'') \kernel(\lambda''-\lambda') 
		= - \theta(\lambda_F - \lvert \lambda \rvert) \int_{-\textrm{sgn}(\lambda_h)\infty}^{\lambda_h} \dd \lambda'     \kernel(\lambda-\lambda').
\end{align}
A change of variables \(a=\lambda-\lambda'\) results in:
\begin{align}
	 - \theta(\lambda_F - \lvert \lambda \rvert) \int_{-\textrm{sgn}(\lambda_h)\infty}^{\lambda_h} \dd \lambda'     \kernel(\lambda-\lambda') = 
	  \theta(\lambda_F - \lvert \lambda \rvert) \int_{\lambda+\textrm{sgn}(\lambda_h)\infty}^{\lambda-\lambda_h} \dd a \, \kernel(a).
\end{align}
Since we again assume \(\lambda\) to be finite, the lower bound of the integral reduces to \(\textrm{sgn}(\lambda_h)\infty\).
Because both \({\lvert \lambda_h \rvert\ < \lambda_F}\) and \({\lvert\lambda\rvert < \lambda_F}\) (due to the Heaviside function), we can no longer make statements about the sign of \(\lambda-\lambda_h\), and have to be a bit more general in our expression.
We get\todo{Check whether particle and hole expressions are the same.}
\begin{align}
	  \theta(\lambda_F - \lvert \lambda \rvert) \int_{\textrm{sgn}(\lambda_h)\infty}^{\lambda-\lambda_h} \dd a \, \kernel(a) &=
	  \frac{c}{\pi}\theta(\lambda_F - \lvert \lambda \rvert) \int_{\textrm{sgn}(\lambda_h)\infty}^{\lambda-\lambda_h} \dd a \frac{1}{c^2 + a^2}\\
	  &= \frac{1}{\pi}\theta(\lambda_F - \lvert \lambda \rvert) \left[ \atan(\frac{\lambda-\lambda_h}{c}) - \textrm{sgn}(\lambda_h)\frac{\pi}{2}\right].
\end{align}

For hole-like excitations the part of the displacement function involving \(\inversetruncc(\lambda,\lambda')\) becomes
\begin{align}
	 - \int_{-\textrm{sgn}(\lambda_h)\infty}^{\lambda_h} &\dd \lambda' \int_{-\lambda_F}^{\lambda_F} \dd \lambda''  \inversetruncc^{(F)}(\lambda,\lambda'') \kernel(\lambda''-\lambda')\\
	 &= - \int_{-\lambda_F}^{\lambda_F} \dd \lambda''  \inversetruncc^{(F)}(\lambda,\lambda'')\left[ \int_{-\textrm{sgn}(\lambda_h)\infty}^{\lambda_h} \dd \lambda' \kernel(\lambda''-\lambda')\right]\\
	 &= \int_{-\lambda_F}^{\lambda_F} \dd \lambda''  \inversetruncc^{(F)}(\lambda,\lambda'')\left[ \int_{\textrm{sgn}(\lambda_h)\infty}^{\lambda''-\lambda_h} \dd a \, \kernel(a)\right]\\
	 &= \frac{1}{\pi}\int_{-\lambda_F}^{\lambda_F} \dd  \lambda''  \inversetruncc^{(F)}(\lambda,\lambda'')\left[ \atan(\frac{\lambda''-\lambda_h}{c}) - \textrm{sgn}(\lambda_h)\frac{\pi}{2}\right].
\end{align}

The expression for the displacement of type II excitations is thus
\begin{multline}
	D_h(\lambda, \lambda_h) = \frac{1}{\pi}\theta(\lambda_F - \lvert \lambda \rvert) \left[ \atan(\frac{\lambda-\lambda_h}{c}) - \textrm{sgn}(\lambda_h)\frac{\pi}{2}\right] \\+
	\frac{1}{\pi}\int_{-\lambda_F}^{\lambda_F} \dd  \lambda''  \inversetruncc^{(F)}(\lambda,\lambda'')\left[ \atan(\frac{\lambda''-\lambda_h}{c}) - \textrm{sgn}(\lambda_h)\frac{\pi}{2}\right].
\end{multline}

\subsection{Calculating \(\inversetruncc^{(F)}\)}

In the above expressions we still have \(\inversetruncc^{(F)}\), for which there is no closed form solution, and which thus has to be numerically evaluated.

The definition of \(\inversetruncc^{(F)}\) is \cite{tofind}
\begin{equation}
	\left(1 + \inversetruncc^{(F)}\right) * \left(1 - \kernel^{(F)}\right)(\lambda,\lambda')=\delta(\lambda-\lambda'), \quad \lambda, \lambda' \in \mathbb{R}
\end{equation}
or explictitly
\begin{equation}
	\int_{-\infty}^{\infty} \dd \lambda'' \left( \delta(\lambda-\lambda'') + \inversetruncc^{(F)}(\lambda,\lambda'')\right)\left(\delta(\lambda''-\lambda') - \kernel^{(F)}(\lambda'',\lambda')\right) = \delta(\lambda-\lambda'),
\end{equation}
with  \(\lambda, \lambda', \lambda'' \in \mathbb{R}\).
If we perform the integration, cancel common terms and rearrange, we get the integral equation
\begin{align}
	\inversetruncc^{(F)}(\lambda,\lambda') &= \int_{-\infty}^{\infty} \inversetruncc^{(F)}(\lambda,\lambda'') \kernel^{(F)}(\lambda'',\lambda') + \kernel^{(F)}(\lambda,\lambda')\\
	&= \theta(\lambda_F - \lvert\lambda'\rvert) \int_{-\lambda_F}^{\lambda_F} \dd \lambda'' \inversetruncc^{(F)}(\lambda,\lambda'') \kernel(\lambda''-\lambda') + \kernel^{(F)}(\lambda,\lambda')
\end{align}

\subsection{Excited states}

For excited states there is no longer a Fermi level \cite{tofind} (or the Fermi level goes to \(\infty\)), meaning that the truncated kernel \(\kernel^{(F)}\) becomes equal to the regular Lieb-Liniger kernel~\(\kernel\).
The displacement function for type I and II excitations are respectively \cite{tofind}
\begin{align}
	D_p(\lambda, \lambda_p) = - \int_{\lambda_p}^{\textrm{sgn}(\lambda_p)\infty} \dd \lambda' \int_{-\infty}^{\infty} \dd  \lambda'' \left[\delta(\lambda-\lambda'') + \inversetruncc(\lambda-\lambda'') \right]\kernel(\lambda''-\lambda'),
\end{align}
and
\begin{align}
	D_h(\lambda, \lambda_h) = - \int_{-\textrm{sgn}(\lambda_h)\infty}^{\lambda_h} \dd \lambda' \int_{-\infty}^{\infty} \dd \lambda'' \left[\delta(\lambda-\lambda'') + \inversetruncc(\lambda - \lambda'') \right]\kernel(\lambda''-\lambda'),
\end{align}
where \(\inversetruncc\) is the inverse of of the kernel \(\kernel\), defined by \cite{tofind}
\begin{align}
	(1+\inversetruncc)*(1-\kernel)(\lambda) = \delta(\lambda),
\end{align}
or explicitly
\begin{align}
	\int_{\infty}^{\infty} \dd \lambda' \left[ \delta(\lambda-\lambda') + \inversetruncc(\lambda-\lambda') \right] \left[ \delta(\lambda') - \kernel(\lambda')\right] = \delta(\lambda), \quad \lambda, \lambda' \in \mathbb{R}.
\end{align}
If we perform the integral and rearrange, we get an expression which simplifies the displacement functions:
\begin{align}\label{eq:kernelinversion}
	\int_{\infty}^{\infty} \dd \lambda' \inversetruncc(\lambda-\lambda') \kernel(\lambda') = -\kernel(\lambda) + \inversetruncc(\lambda).
\end{align}

With this in hand, we have for particles:
\begin{align}
	D_p(\lambda, \lambda_p) &= - \int_{\lambda_p}^{\textrm{sgn}(\lambda_p)\infty} \dd \lambda' \int_{-\infty}^{\infty} \dd  \lambda'' \left[\delta(\lambda-\lambda'') + \inversetruncc(\lambda-\lambda'') \right]\kernel(\lambda''-\lambda')\\
	&= - \int_{\lambda_p}^{\textrm{sgn}(\lambda_p)\infty} \dd \lambda' \left[\kernel(\lambda-\lambda') + \int_{-\infty}^{\infty} \dd  \lambda'' \inversetruncc(\lambda-\lambda'') \kernel(\lambda''-\lambda')\right]\\
	&= - \int_{\lambda_p}^{\textrm{sgn}(\lambda_p)\infty} \dd \lambda' \left[\kernel(\lambda-\lambda') - \kernel(\lambda-\lambda') + \inversetruncc(\lambda-\lambda')\right]\\
	&= - \int_{\lambda_p}^{\textrm{sgn}(\lambda_p)\infty} \dd \lambda' \inversetruncc(\lambda-\lambda')\\
	&=  \int_{\lambda-\lambda_p}^{-\textrm{sgn}(\lambda_p)\infty} \dd a \, \inversetruncc(a),
\end{align}
where we used \cref{eq:kernelinversion} in going from the second to the third line, and made the substitution \(a = \lambda-\lambda'\) in the last line.
A very similair derivation for type II excitations leads to
\begin{align}
	D_h(\lambda,\lambda_h) = \int_{\textrm{sgn}(\lambda_h)\infty}^{\lambda-\lambda_h} \dd a\, \inversetruncc(a).
\end{align}

\chapter{Results}

\chapter{Discussion and conclusion}


\bibliographystyle{SciPost_bibstyle}
\bibliography{master_thesis_references.bib}

\end{document}
